\documentclass[12pt,a4paper]{article}
\title{Common Template}
\author{Claude Lu}

\usepackage{xcolor}     
\usepackage{float}
\usepackage{graphicx}
\usepackage{enumerate}
\usepackage{enumitem} %允許自定義bullet
\usepackage{amsmath, amssymb, amsfonts} %basic package for math expressions.
\usepackage{listings}
\usepackage[top=1in, bottom=1in, left=1in, right=1in]{geometry}
%表格必用
\usepackage{multirow}
\usepackage{hhline, booktabs}
%enable hyperlink
\usepackage{hyperref}
\usepackage{array}

%enable lstlisting
\usepackage{listings}

\usepackage{framed}

%自定義顏色
\definecolor{rublue}{HTML}{0036A7}
\definecolor{rured}{HTML}{D62718}
\definecolor{vscodegreen}{HTML}{4EC9B0}
\definecolor{vscodeblack}{HTML}{1E1E1E}
%\definecolor{vscodeblack}{HTML}{4D4D4D}
\definecolor{vscodeyellow}{HTML}{DCDCAA}
\definecolor{vscodeblue}{HTML}{569CD6}
%自定義命令
\newcommand{\bluetexttt}[1]{\textcolor{blue}{\texttt{#1}}}
\newcommand{\redtexttt}[1]{\textcolor{red}{\texttt{#1}}}
\newcommand{\redtext}[1]{\textcolor{red}{#1}}

\newcommand{\rt}[1]{\textcolor{red}{#1}}
\newcommand{\bt}[1]{\textcolor{blue}{#1}}
\newcommand{\vsgt}[1]{\textcolor{vscodegreen}{#1}}
\newcommand{\vsft}[1]{\textcolor{vscodeyellow}{#1}}
\newcommand{\vsbt}[1]{\textcolor{vscodeblue}{#1}}

\newcommand{\struct}[1]{\colorbox{vscodeblack}{\vsgt{#1}}}
\newcommand{\function}[1]{\colorbox{vscodeblack}{\vsft{#1}}}
\newcommand{\macro}[1]{\colorbox{vscodeblack}{\vsbt{#1}}}


\newcommand{\bs}[1]{\boldsymbol{#1}}


\lstset{
	basicstyle=\ttfamily, %use typewriter font
	keywordstyle=\color{blue},%關鍵字顏色
	commentstyle=\color{teal}, %註釋顏色 
	stringstyle=\color{red}, % Strings in red
	tabsize=4, % Set tab size to 4 spaces
	showspaces=false, % Do not show spaces
	showstringspaces=true, % Do not show string spaces
	breaklines=true, %自動換行
	frame=single %在代碼外加上外框
}
\begin{document}
% show manual
\tableofcontents
\maketitle

\section{Math Expressions}
\subsection{Display Mode}
superscipts: $$2x^3$$
$$2x^{2x+3}$$
subscripts: $$2x_{3}$$
$$x_{123}$$
$$a_0, a_1, a_2, \ldots, a_{100}$$
Greek letters
$$\pi$$
$$\Pi$$
Trig functions
$$y=\sin x$$ %always add '\' before trig functions
$$y=\sin^{-1}\theta$$	
$$y=\arcsin x$$
Log functions
$$y=\log_5 x$$
$$y=\ln x$$
Roots
$$\sqrt{2}$$
$$\sqrt[3]{27}$$
$$\sqrt{x^2+y^2}$$
Fractions
$$\frac{2}{3}$$
Mathbb
$$\mathbb{R}$$
Parentheses with larger size: $$2\left(\frac{1}{2}\right)$$ %use \left( \right) to address 
brackets
$$\left( \frac{1}{2} \right)$$  	
$$\left[ \frac{1}{2} \right]$$
$$\left\{ \frac{1}{2} \right\}$$ %打大括號比較特別
$$\left \langle \frac{1}{2} \right \rangle$$
$$\left | \frac{1}{2} \right |$$
$$\left. \frac{dy}{dx}\right|_{x=1}$$ %use . to not display half of brackets
$$\left( \frac{1}{1+\left( \frac{1}{2} \right)} \right)$$ %()will adjust size depends on content
\section{Notations for Calculus}
The function $f(x)=(x-3)^2 + \frac{1}{2}$ has domain $\mathrm{D}_f:(-\infty,\infty)$ and range $\mathrm{R}_f:\left [\frac{1}{2},\infty\right )$\\

$\lim \limits_{x \to a^-} f(x)$ %use '\to' for arrow

$\lim \limits_{x \to a} \frac{f(x)-f(a)}{x-a}=f'(a)$

$\int \sin xdx$

%use \displaystyle for larger integral sign
$\displaystyle{\int sinx \,dx}$ %use \, for space in math mode 

$\displaystyle{\int \limits_{a}^{b}x^{2} \,dx=\left [ \frac{x^{3}}{3} \right ]_{a}^{b}}$

$\displaystyle{\int \limits_{t_{0}}^{t}e^{-\alpha(t-\tau)}f(\tau)\,d\tau}$

$\displaystyle{e^{-\alpha t}\int \limits_{t_{0}}^{t}e^{\alpha t}f(t)\,dt}$

$\displaystyle{\sum \limits_{n=1}^{\infty}ar^{n}=a+ar+ar^{2}+\cdots+ar^{n}}$ %use cdot for 1 dot, cdots for 3 dots

$\vec{v}=v_{1} \vec{i} + v_{2} \vec{j} = \langle v_{1},v_{2} \rangle$
\pagebreak
\section{Table}
\begin{table}[H]
	\centering
	\begin{tabular}{||c|c|c||}
		\hhline{#===#}
		% Use multicolumn to merge cells in the same row
		% \multicolumn{number_of_columns}{alignment}{content}
		\multicolumn{3}{||c||}{Graph Border Encoding}\\ \hline %hline for a line
		Bit & plot & splot\\ 
		\hhline{#===#}
		1 & bottom & bottom left front\\ \hline
		2 & left & bottom left back\\ \hline
		4 & top & botton right front\\ \hline
		8 & right & bottom right back\\ \hline
		% use multirow to merge cells in the same column
		16 & \multirow{8}{*}{no effect} & left vertical\\ \cline{1-1} \cline{3-3}
		% when \hline does not fit, use cline to assign line for each ceel
		32 & & back vertical\\ \cline{1-1} \cline{3-3}
		64 & & right vertical\\ \cline{1-1} \cline{3-3}
		128 & & front vertical\\ \cline{1-1} \cline{3-3}
		256 & & top left back\\ \cline{1-1} \cline{3-3}
		512 & & top right back\\ \cline{1-1} \cline{3-3}
		1024 & & top left front\\ \cline{1-1} \cline{3-3}
		2048 & & top right front\\ \cline{1-1} \cline{3-3}
		\hhline{#===#}
	\end{tabular}
	\label{tab:table_template}
	\caption{Table Template}
\end{table}

	The position of table is decided by compiler, unless you specify where it should be.\\[6pt]
\begin{tabular}{|c||c|c|c|c|c|} %column numbers and alignment c=center, l=left, r=right
	\hline %add horizontal line 
	1 &2 &3 &4 &5 &6\\ \hline
	$f(x)$&10&5&5&4&5\\ \hline
\end{tabular}

\vspace{1cm} %set vertical space

\begin{table}[H] %must use 'float' package
	\centering %center the table
	\def\arraystrech{1.5} %拉大表格
	\begin{tabular}{|cccccc|}
		\hline 
		1 &2 &3 &4 &5 &6\\ \hline 
		1 &2 &3 &4 &5 &6\\ \hline
	\end{tabular}
	\caption{Test Table} %add caption to table
\end{table}
\vspace{1cm}
\begin{table}[H]
	\centering
	\begin{tabular}{|l|p{5cm}|} %use p{width} for paragraph instead of l, r, or c for paragraph and assign width
		\hline
		$f(x)$ & $f'(x)$ \\ \hline
		$x > 0$ & The function $f(x)$ is increasing. The function $f(x)$ is increasing. The function $f(x)$ is increasing.\\
		\hline
	\end{tabular}
\end{table}
\pagebreak

\section{list}
\subsection{basic options}
\begin{enumerate}
	\item pencil
	\item calculator
	\item ruler
	\begin{enumerate}
		\item steel ruler
		\begin{enumerate}
			\item long steel ruler
			\item short steel ruler
		\end{enumerate}
	\end{enumerate}
\end{enumerate}

%\begin{enumerate}[A.] %use package enumerate to assign symbol for enumeration
%	\item pencil
%	\item calculator
%	\item ruler
%\end{enumerate}
\vspace{1cm}
\begin{enumerate} \setcounter{enumi}{5} %set the start number of enumeration
	\item pencil
	\item calculator
	\item ruler
\end{enumerate}
%\vspace{1cm}
Bullet list
\begin{itemize} %bullet list
	\item pencil
	\item calculator
	\item ruler
\end{itemize}
Custom bullet for enumerate environment
\begin{enumerate} %to cancel the number, use []
	\item[one] pencil %for customization
	\item[two] calculator
	\item[] ruler
\end{enumerate}
\pagebreak

\section{text document formatting}
\subsection{font size and style}
This will produce \textit{italicized} text.

This will produce \textbf{bold face} text.

This will produce \textsc{small caps} text.

This will produce \texttt{typewriter font} text.

Please visit Michelle Krummel's website at \url{http://michellekrummel.com} %use hyperref for hyperlink
Use href to set text you want to display for hyperlink like \href{http://michellekrummel.com}{This}
\vspace{1cm}
This will produce \begin{large}large font\end{large}

This will produce \begin{Large}Large font\end{Large}

This will produce \begin{huge}huge font\end{huge}

This will produce \begin{Huge}Huge font\end{Huge}

This will produce \begin{normalsize}normal font\end{normalsize}

This will produce \begin{small}small font\end{small}

This will produce \begin{footnotesize}footnotesize font\end{footnotesize}

This will produce \begin{scriptsize}scriptsize font\end{scriptsize}

This will produce \begin{tiny}tiny font\end{tiny}
\subsection{text color}
Define textcolor by yourself
\begin{verbatim}
	\definecolor{rublue}{HTML}{0036A7} %self-defined color
	\newcommand{\bluetext}[1]{\textcolor{blue}{#1}} %for faster colouring
\end{verbatim}
\textcolor{red}{This can change color of text} %可以改變字體顏色
\subsection{alignment}
\begin{center}
	This line is centered
\end{center}

\begin{flushleft}
	This line is left-justified
\end{flushleft}

\begin{flushright}
	This line is right-justified
\end{flushright}
\subsection{verbatim}
Use verbatim to ``display normally'', ignoring all latex commands. Useful when typing contents containing programming language. Beware that verbatim environment ignores \textbf{tab}, but preserves \textbf{space}.
\begin{verbatim}
	#include<iostream>
	int main(){
		std::out << Hello World << std::endl;
		return 0;
	}
\end{verbatim}
%\center
%The following content is all centered
\subsection{listings}
ChatGPT strongly recommend using \texttt{listings} for program-related content. To use it, you have to include use package \texttt{listings} adn use the command \texttt{lstset} first: 
\begin{verbatim}
	\lstset{
		basicstyle=\ttfamily, %use typewriter font
		keywordstyle=\color{*}
		commentstyle=\color{*}, 
		stringstyle=\color{*}, % Strings in red
		tabsize=4, % Set tab size to 4 spaces
		showspaces=false, % Do not show spaces
		showstringspaces=false, % Do not show string spaces
		breaklines=true, % Automatic line breaking
		frame=single % Add a frame around the code
	}	    
\end{verbatim}
\lstset{
	basicstyle=\ttfamily, %use typewriter font
	keywordstyle=\color{blue},%關鍵字顏色
	commentstyle=\color{green}, %註釋顏色 
	stringstyle=\color{red}, % Strings in red
	tabsize=4, % Set tab size to 4 spaces
	showspaces=false, % Do not show spaces
	showstringspaces=false, % Do not show string spaces
	breaklines=true, %自動換行
	frame=single %在代碼外加上外框
}
Then you can use display your code in a ``IDE'' way
\begin{lstlisting}[language=Python]
	def example_function():
	# This is a comment
	print("Hello, world!") # Print statement
	
	if True:
	print("Condition met")
	else:
	print("Condition not met")
\end{lstlisting}

\section{Picture}
	\begin{figure}[H] %h stands for here, b stands for bottom
	\centering
	\includegraphics[width=10cm]{./Pictures/example} %use package graphicx
	\caption{picture}		
\end{figure}


\end{document}