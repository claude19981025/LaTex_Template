\documentclass{article}
\usepackage[utf8]{inputenc}   % 支持 UTF-8 編碼
\usepackage{amsmath, amssymb} % 支持數學符號
\usepackage{graphicx}         % 插入圖片
\usepackage{hyperref}         % 超鏈接
\usepackage{caption}          % 圖表標題
\usepackage{subcaption}       % 子圖表
\usepackage{listings}         % 代碼高亮
\usepackage{xcolor}           % 顏色
\usepackage{geometry}         % 頁面設置
\usepackage{fancyhdr}         % 頁眉頁腳
\usepackage{booktabs}         % 美化表格
\usepackage{multicol}         % 多欄排版
\usepackage{tikz}             % 繪圖
\usepackage{enumitem}         % 自定義列表


%中文輸入配置模塊
%如果報錯不能編譯,請至Options -> Configure TeXstudio後,選Build,將Default Compiler設置爲支持xeCJK包的XeLatex或LuaLaTeX



\ifXeTeX	%若使用XeTex編譯器
\usepackage{xeCJK}	% 支持中文輸入
\usepackage{fontspec}	% 允許更換字體
\setmainfont[Path=./fonts/]{TW-Kai-98-1.ttf} %楷體
%\setmainfont[Path=./fonts/]{TW-Sung-98-1.ttf} %宋體
%\newfontfamily\song{SimSun}
\fi

\ifLuaTeX
\usepackage{luatexja-fontspec}	% LuaLaTeX支持的中文包
\usepackage{fontspec}	% 允許更換字體
\setmainfont[Path=./fonts/]{TW-Sung-98-1.ttf} %宋體
\fi


% 頁面設置
\geometry{a4paper, margin=1in}

% 頁眉頁腳設置
\pagestyle{fancy}
\fancyhf{}
\fancyhead[L]{Your Name} %[位置][內容] 位置:E(偶數頁)O(奇數頁)LCR
\fancyhead[R]{\thepage}

% 代碼高亮設置
\lstset{
	basicstyle=\ttfamily\small,
	keywordstyle=\color{blue}\bfseries,
	commentstyle=\color{green},
	stringstyle=\color{red},
	showstringspaces=false,
	frame=single,
	numbers=left,
	numberstyle=\tiny\color{gray},
	breaklines=true,
	backgroundcolor=\color{lightgray!20},
	captionpos=b
}

% 用於縮短高亮指令
\newcommand{\code}[1]{\texttt{#1}}
\newcommand{\todo}[1]{\textcolor{red}{TODO: #1}}

\title{A4繁體中文模板}
\author{盧永鈞}
\date{\today}

\begin{document}
	
\maketitle % 打印作者姓名,編輯日期

%\begin{abstract}
	
%\end{abstract}
	
\tableofcontents %打印目錄
	
\section{簡介}
這是簡介部分,簡要介紹文檔的背景和目的。Enslish font is ok
	
\section{圖表示例}
	
\subsection{插入圖片}
\begin{figure}[h]
	\centering
	\includegraphics[width=0.5\textwidth]{example-image} % 替換為你的圖片文件
	\caption{這是一個示例圖片}
	\label{fig:example}
\end{figure}
	
\subsection{插入表格}
\begin{table}[h]
	\centering
	\begin{tabular}{|c|c|c|}
		\hline
		A & B & C \\
		\hline
		1 & 2 & 3 \\
		4 & 5 & 6 \\
		7 & 8 & 9 \\
		\hline
	\end{tabular}
	\caption{這是一個示例表格}
	\label{tab:example}
\end{table}
	
\subsection{並排圖表}
\begin{figure}[h]
	\centering
	\begin{subfigure}{0.45\textwidth}
		\centering
		\includegraphics[width=\textwidth]{example-image-a} % 替換為你的圖片文件
		\caption{子圖 A}
	\end{subfigure}
	\hfill
	\begin{subfigure}{0.45\textwidth}
		\centering
		\includegraphics[width=\textwidth]{example-image-b} % 替換為你的圖片文件
		\caption{子圖 B}
	\end{subfigure}
	\caption{並排圖表示例}
	\label{fig:sidebyside}
\end{figure}
	
\section{代碼示例}
\begin{lstlisting}[language=Python, caption=Python 代碼示例]
	def hello_world():
	print("Hello, world!")
	
	hello_world()
\end{lstlisting}
	
\section{結論}
結論打在這裏
	
\end{document}
